\chapter{Conclusion}

In this project we explored the implementation of quantum information functions in two popular programming languages MATLAB and Python. We explored quantum information toolboxes available for both of them. After studying the advantages and limitations of both we decided to settle on Python and the Quantum Toolbox in Python (QuTiP). The successful implementation of the target functions indicates that QuTiP can serve as an excellent base for writing a quantum information toolbox.
\par MATLAB has long been dominant in the scientific programming community due to its ease of use and has gained a lot of momentum as people wrote different science and engineering toolboxes for it. Hence it has collected a huge number of very useful libraries. Its easy syntax and the availability of lots of toolboxes make it a popular choice in academia. However its license is restrictive and very expensive which makes it difficult for students and researchers to acquire a copy for their work. This also makes it an unpopular choice in industry where tough competition demands that costs be kept as low as possible.
\par Python with NumPy and SciPy on the other hand has the advantage of being completely free and open source. This makes it very attractive to the scientific programming community since science by its nature is an open process where people freely build on top of other people's work. The syntax is clearer and easier to learn. It has access to a large number of other general purpose libraries that can be used along with the scientific libraries to write more powerful programs. Python is more cross-platform than MATLAB, running on desktops, servers, mobile devices and embedded systems. This along with its non-restrictive license makes it very useful in the industry. All these advantages have been making Python increasingly popular in the scientific programming community and the number of libraries and toolboxes available for science and engineering has been growing fast.
\par However, quantum information toolboxes available for Python are fewer than there are for MATLAB \cite{quantikiqcsimulators}. The toolbox we used, QuTiP, is an excellent simulator for open quantum systems but was never designed specifically for quantum information. Hence it lacks many functions needed in quantum information theory. However we also saw that these functions can be added to QuTiP very easily and it incorporates them very well into its infrastructure. Building on top of QuTiP's well established base not only benefits the end product by eliminating a lot of development overhead, it also makes it possible for the newer functionality to be incorporated into QuTiP itself, increasing the scope of the library.
\par Over all, while writing new quantum information functions for Python is an investment of time and resources, it is an investment well worth making. Considering the advantages Python provides over MATLAB, this is the direction that will provide better results in the long term. The functions need not be written from scratch either. Functions from other toolboxes like QLib and QETLAB can be studied as a reference. Of course, this borrowing of ideas can go both ways.
\par An effort has been made in designing this document so that in the future more students from the department of physics can base their work on top of this project and add further quantum information functionality to QuTiP. A group of masters students is already working on other projects as part of which more functions will be written. These projects will study GUP-deformed and q-deformed quantum entanglement and the application of tensor networks to quantum information theory.

