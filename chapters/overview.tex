\chapter{Overview of the Project}

\par The development of quantum mechanics in the early twentieth century introduced us to a whole new way of thinking about the universe. Moreover, it also introduced some extremely strange and non intuitive ideas. One such idea is the concept of non-locality. The phenomenon of quantum entanglement seemed so strange to the intellectual giants of that time, that they ended up calling it “spooky action at a distance”. Although later developments in quantum information theory showed this spooky action can be exploited for novel applications in communication. Today researchers are looking at quantum entanglement as an important resource in developing communication systems that nobody would have thought of fifty years earlier.
\par This project is primarily concerned with the phenomenon of quantum entanglement and the different ways to detect and quantify it. We have carried out some simulations of quantum information systems and implemented various tests and measures of quantum entanglement, along with some other quantum information functions, in the Quantum Toolbox in Python (QuTiP). Besides Python, we have also explored MATLAB and the quantum information toolboxes available for both of them.
\par In particular this project is concerned with \textit{adding} quantum information functionality to the QuTiP toolbox. QuTiP is an excellent simulator for open quantum systems but is currently lacking in terms of quantum information functionality. This project hopes to improves this. For now entanglement tests and measures are focused on more. In the future, this work can be extended to add even more QI functionality to QuTiP.
\par Here is a brief overview of this report.
\begin{enumerate}
  \item Chapter one introduces the basics of classical information theory.
  \item Chapter two provides a review of basic quantum mechanics.
  \item Chapter three extends the concepts of classical information theory to quantum information theory.
  \item Chapter four introduces the idea of computer simulations of physical systems and discusses some tools that we have available in the scope of this project. In this chapter we introduce MATLAB and Python, two popular programming languages in academia. We take a look at various toolboxes available for both to do quantum information simulations and research.
  \item Chapter five discusses the basics of QuTiP, an open quantum systems simulator for Python, and demonstrates how it can be used for solving quantum information problems.
  \item Chapter six explains in detail the code developed in this project. It details the implementation of the new quantum information functions in QuTiP, focusing mostly on entanglement tests and measures. It also demonstrates some basic simulations involving entangled states, applying the new functions and measures to them.
\end{enumerate}
We hope that this project will spark some interest in computational physics in our department and in the future more students will step forward to contribute in the development of more scientific programming libraries.

